\documentclass{article}


\begin{document}
\title{Sphere Surface Area Explanation}
\author{Jayke Nguyen}
\maketitle
It can be shown that the total surface area of a sphere can be calculated by:
$$\int_{0}^{2\pi} d\phi \int_{0}^{\pi} sin\theta d\theta = 4\pi $$

For some arbitrary portion surface area we can calculate the surface area by: 
$$\int_{a}^{b} d\phi \int_{c}^{d} sin\theta d\theta = Area $$
Simplifying this becomes:
$$\int_{a}^{b} d\phi \int_{c}^{d} sin\theta d\theta = (b-a)(-cos(d)+cos(c)) = (b-a) (cos(c)-cos(d))$$

Additionally, we know that:
$$ b-a = d-c = \pi/r = constant $$
Where $r$ denotes the resolution that is user-defined. This comes from the fact that the for-loops are defined as the following:

\begin{verbatim}
for(i = 0; i < ANGULAR_RES; i++){
    sph.theta = double(PI)/double(ANGULAR_RES) * double(i);
    for(j = 0; j < ANGULAR_RES*2; j++){
        sph.phi = double(PI*2.0)/double(ANGULAR_RES*2.0) * double(j);
\end{verbatim}

Therefore, we can rearrange the above equation into an equation only dependent on d.
$$ (b-a) (cos(c)-cos(d)) = \frac{\pi}{r} \Bigg(cos\Big(d-\frac{\pi}{r}\Big) - cos(d)\Bigg) = Area$$

Where d is the variable that we are looping over in the $\theta$ for-loop.
So, in the code, I continually add this area until we have integrated over the whole sphere. I implement an "area-counter" variable, that successively adds these areas for only the angles that fulfill the various criterion for a pair match.

\end{document}